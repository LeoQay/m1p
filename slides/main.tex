\documentclass[10pt,pdf,hyperref={unicode}]{beamer}

\mode<presentation>
{
	\usetheme{boxes}
	\beamertemplatenavigationsymbolsempty
	
	\setbeamertemplate{footline}[page number]
	\setbeamersize{text margin left=0.5em, text margin right=0.5em}
}

\usepackage[utf8]{inputenc}
\usepackage[english, russian]{babel}
\usepackage{bm}
\usepackage{multirow}
\usepackage{ragged2e}
\usepackage{indentfirst}
\usepackage{multicol}
\usepackage{subfig}
\usepackage{amsmath,amssymb}
\usepackage{enumerate}
\usepackage{mathtools}
\usepackage{comment}
\usepackage{multicol}

\usepackage[all]{xy}

\usepackage{tikz}
\usetikzlibrary{positioning,arrows}

\tikzstyle{name} = [parameters]
\definecolor{name}{rgb}{0.5,0.5,0.5}

\usepackage{caption}
\captionsetup{skip=0pt,belowskip=0pt}

\newtheorem{rustheorem}{Теорема}
\newtheorem{russtatement}{Утверждение}
\newtheorem{rusdefinition}{Определение}

% colors
\definecolor{darkgreen}{rgb}{0.0, 0.2, 0.13}
\definecolor{darkcyan}{rgb}{0.0, 0.55, 0.55}

\AtBeginEnvironment{figure}{\setcounter{subfigure}{0}}

\captionsetup[subfloat]{labelformat=empty}



%---------------------------------------------------------------------------------------------------------
\begin{document}
	
	\begin{frame}
		\titlepage
	\end{frame}
	
	%----------------------------------------------------------------------------------------------------------
	\section{Слайд об исследованиях}
	\begin{frame}{Слайд об исследованиях}
		\bigskip
		Исследуется проблема \ldots.
		\begin{block}{Цель исследования~---}
			предложить метод \ldots.
		\end{block}
		\begin{block}{Требуется предложить}
			\justifying
			\begin{enumerate}[1)]
				\item метод \ldots,
				\item метод \ldots,
				\item метод \ldots.
			\end{enumerate}
		\end{block}
		\begin{block}{Решение}
			Для \ldots.
		\end{block}
	\end{frame}
	
	%---------------------------------------------------------------------------------------------------------
	\section{Постановка задачи}
	\begin{frame}{Постановка задачи}
		
		Имеется выборка $\tilde{S} = \{ (y_j, x_j, G_1(x_j), G_2(x_j)), j = \overline{1, m} \}$, где
		\begin{itemize}
			\item $y_j$ - значение переменной $Y$ для объекта с номером $j$
			\item $x_j = (x_{j1}, \dots, x_{jn})$ - вектор значений признаков $X_1, \dots, X_n$ для объекта с номером $j$
			\item $G_1(x_j)$ - значение функции $G_1$ в точке $x_j$
			\item $G_2(x_j)$ - значение функции $G_2$ в точке $x_j$
		\end{itemize}
		
		Предлагается построить дерево $T(x)$, для которого достигается минимум функционала:
		$$ \Phi(\tilde{S}, T) = \Sigma_{j=1}^{m} \{ \gamma_1 [T(x_j) - y_j]^2 + \gamma_2 [T(x_j) - G_2(x_j)]^2 - \mu [T(x_j) - G_1(x_j)]^2   \}  $$
		где $\gamma_1 + \gamma_2 = 1; \quad \gamma_1, \gamma_2, \mu \in [0, 1] $
		
		\bigskip
		
		Вместо независимой и параллельной генерации деревьев,
		будем на каждом шагу добавлять дерево, сильно отличающееся от уже
		созданного ансамбля, с помощью специального функционала,
		учитывающего ответы предыдущих моделей.
		

	\end{frame}
	

	

	

	
\end{document} 